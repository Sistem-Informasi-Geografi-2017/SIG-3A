\section{MUH. RIFKY PRANANDA (1174017)}
\begin{figure}[H]
	\includegraphics[width=4cm]{figures/1174017/gis1.JPG}
	\centering
	\caption{Contoh gambar.}
\end{figure}
\subsection{Pengertian}
Geografi adalah ilmu yang mempelajari tentang lokasi serta persamaan dan perbedaan variasi keruangan atas fenomena fisik, dan manusia di atas permukaan bumi. Kata geografi berasal dari Bahasa Yunani yaitu gêo "Bumi", dan graphein "tulisan" atau "menjelaskan". Para sarjana, praktisi, atau penulis di bidang geografi disebut geograf atau geografer.Geografi juga merupakan nama judul buku bersejarah pada subjek ini, yang terkenal adalah Geographia tulisan Klaudios Ptolemaios pada abad kedua.Geografi lebih dari sekadar kartografi, studi tentang peta. Geografi tidak hanya menjawab apa, dan di mana di atas muka bumi, tapi juga mengapa di situ, dan tidak di tempat lainnya, kadang diartikan dengan "lokasi pada ruang." Geografi mempelajari hal ini, baik yang disebabkan oleh alam atau manusia. Juga mempelajari akibat yang disebabkan dari perbedaan yang terjadi itu.
Sistem informasi geografis (SIG) adalah sistem yang didesain untuk menangkap, menyimpan, mengolah, menganalisa, serta mempresentasikan data spasial.Dengan menghubungkan berbagai jenis data yang awalnya dianggap tidak berhubungan, SIG dapat membantu manusia dalam berbagai aspek pekerjaan. Proses menghubungkan ini umumnya dilakukan dalam konteks lokasi (spasial) dan waktu (temporal) yang sama.
\begin{figure}[H]
	\includegraphics[width=4cm]{figures/1174017/gis.JPG}
	\centering
	\caption{Contoh gambar.}
\end{figure}
\subsection{Sejarah}
Pada awalnya, peta hanya memiliki satu atau dua informasi saja didalamnya, sehingga jika seorang analis ingin mendapatkan informasi tambahan, dia harus melakukan overlay.Salah satu proses overlay pertama yang juga dianggap sebagai penggunaan analisa spasial pertama secara sukses adalah oleh John Snow di London pada tahun 1854. Dia melakukan pemetaan terhadap lokasi orang-orang yang mengidap penyakit cholera dan menghubungkannya dengan peta penyediaan air minum London.Pada awal abad ke-20, penggunaan teknik photozincography mulai meluas. Teknik ini memungkinkan peta terdiri dari beberapa layer yang nantinnya dapat diubah secara mandiri dari layer lainnya. Hal ini berguna untuk melakukan overlay dan analisa peta sesuai dengan layer yang dibutuhkan.Pada teknik ini, layer-layer yang ada dibuat dari film plastik atau lapisan kertas kalkir sehingga dapat digabungkan menjadi satu peta besar. Namun, teknik ini belum dapat dianggap sebagai SIG karena tidak terdapat database yang menghubungkan peta-peta tersebut.Pada tahun 1960, Canada mengembangkan sistem SIG pertamanya yang disebut CGIS atau Canada Geographic Information System. Sistem ini digunakan untuk menyimpan, mengolah, dan menganalisa informasi yang dimiliki oleh badan pertanahan Canada.
\begin{figure}[H]
	\includegraphics[width=4cm]{figures/1174017/gis2.JPG}
	\centering
	\caption{Contoh gambar.}
\end{figure}
\subsection{Koordinat}
Peta, Proyeksi Peta, Sistem Koordinat, Survei dan GPS 
   Data spasial yang dibutuhkan pada SIG dapat diperoleh dengan berbagai cara. Salah satunya melalui survei dan pemetaan, yaitu penentuan posisi/koordinat di lapangan. Berikut ini akan dijelaskan secara ringkas beberapa hal yang berkaitan dengan posisi/koordinat serta metode-metode untuk mendapatkan informasi posisi tersebut di lapangan
\subsubsection{peta}
Peta adalah gambaran sebagian atau seluruh muka bumi baik yang terletak di atas maupun di bawah permukaan dan disajikan pada bidang datar pada skala dan proyeksi tertentu (secara matematis). Karena dibatasi oleh skala dan proyeksi maka peta tidak akan pernah selengkap dan sedetail aslinya (bumi). Untuk itu diperlukan penyederhanaan dan pemilihan unsur yang akan ditampilkan pada peta.
\subsubsection{proyeksipeta}
Pada dasarnya bentuk bumi tidak datar, tapi mendekati bulat. Maka untuk menggambarkan sebagian muka bumi untuk kepentingan pembuatan peta, perlu dilakukan langkah-langkah agar bentuk yang mendekati bulat tersebut dapat didatarkan dan distorsinya dapat terkontrol. Caranya dengan melakukan proyeksi ke bidang datar.
\subsubsection{Yang menggunakan bidang proyeksi}
Bidang datar, Bidang kerucut, Bidang silinder
\subsubsection{Proyeksi Universal Tranverse Mercator}
Proyeksi UTM dibuat oleh US Army sekitar tahun 1940-an. Sejak saat itu proyeksi ini menjadi standar untuk pemetaan topografi
\subsubsection{Sistem kordinat UTM}
Untuk menghindari koordinat negatif, dalam proyeksi UTM setiap meridian tengah dalam tiap zone diberi harga 500.000 mT (meter timur). Untuk harga-harga ke arah utara, ekuator dipakai sebagai garis datum dan diberi harga 0 mU (meter utara). Untuk perhitungan ke arah selatan ekuator diberi harga 10.000.000 mU.

   Wilayah Indonesia (90° – 144° BT dan 11° LS – 6° LU) terbagi dalam 9 zone UTM. Artinya, wilayah Indonesia dimulai dari zone 46 sampai zone 54 (meridian sentral 93° – 141° BT).
\subsubsection{Metode penentuan posisi}
Metode penentuan posisi adalah cara untuk mendapatkan informasi koordinat suatu objek di lapangan, contohnya koordinat titik batas, koordinat batas persil tanah dan lain-lain. Metode penentuan posisi dapat dibedakan dalam dua bagian, yaitu metode penentuan posisi terestris dan metode penentuan posisi extra-terestris (satelit).
Pada metode terestris, penentuan posisi titik dilakukan dengan melakukan pengamatan terhadap target atau objek yang terletak di permukaan bumi. Beberapa contoh metode yang umum digunakan adalah:
metode poligon, metode pemikatan ke muka, metode pengikatan ke belakang dan lain lain.
\subsection{Data Geospasial}
\subsubsection{Data spasial}
Data spasial adalah data yang memiliki informasi lokasi pada data tersebut. Informasi lokasi ini umumnya berbentuk sistem koordinat baik itu koordinat geografis ataupun koordinat proyeksi.
Data spasial umumnya digunakan untuk menunjukkan lokasi dari suatu obyek/kenampakan pada dunia nyata.
Terdapat dua jenis data spasial yaitu vektor dan raster. Kedua jenis data ini memiliki perbedaan sifat dan kegunaannya. Oleh karena itu, penggunaannya sangat tergantung dengan kondisi dan hasil yang ingin dicapai.
\subsubsection{Data vektor}
Data vektor adalah data yang direpresentasikan dengan garis, titik, atau polygon. Data vektor dihasilkan dari digitasi data raster ataupun penggambaran langsung obyek dunia nyata pada peta. Oleh karena itu, data vektor lebih sulit dibuat dibandingkan dengan data raster.
\subsubsection{Data Raster}
Data raster adalah data yang direpresentasikan dengan piksel dalam sebuah grafik. Data raster dihasilkan langsung oleh foto udara maupun foto satelit. Oleh karena itu, secara umum, data raster lebih mudah dibuat dibandingkan dengan data vektor.
\subsubsection{Data Aspasial}
Data aspasial adalah data yang tidak memiliki informasi mengenai lokasi data tersebut. Data ini umumnya digunakan untuk membantu menjelaskan informasi yang terkandung pada data spasial.
Contoh data aspasial adalah data atribut suatu obyek. Misal pada suatu peta terdapat titik berwarna hitam pada koordinat tertentu, kita tidak akan tahu titik tersebut bermakna apa tanpa adanya penjelas yaitu legenda. Legenda adalah salah satu contoh data aspasial.
Sistem informasi geografis modern menggunakan data digital untuk proses analisa dan Y666penafsirannya. Berikut ini adalah beberapa metode pengumpulan data untuk analisa SIG.
Digitasi adalah proses mendigitalkan data yang bersifat fisik. Hal ini perlu dilakukan karena mayoritas data perpetaan masih berada dalam bentuk fisik seperti pada lembaran film atau kertas peta.
Proses ini dapat dilakukan dengan cara memasukkan data fisik seperti foto dan peta kedalam mesin, atau men-scan data tersebut dan mendigitasinya secara manual dengan aplikasi.Proses digitasi akan mengubah data raster menjadi data vektor yang dapat diolah dan dianalisa oleh aplikasi SIG.
Survey juga merupakan bagian yang sangat penting dari SIG. Survey meliputi aktivitas pengambilan data langsung di tempatSurvey menghasilkan data yang nantinya dapat langsung dimasukkan kedalam basis data SIG dengan menggunakan coordinate geometry (COGO).Meskipun telah ada teknologi penginderaan jauh, survey masih sangat dibutuhkan dalam proses pemetaan. Hal ini disebabkan oleh keberadaan informasi-informasi detail lokasi yang mungkin tidak dapat ditangkap dan digambarkan oleh penginderaan jauh.
\subsection{https://youtu.be/QSoJBLX-hQ0}
\subsection{Plagiarism}
\begin{figure}[H]
	\includegraphics[width=4cm]{figures/1174017/plagiat.png}
	\centering
	\caption{plgt}
\end{figure}